\documentclass[10pt, conference, compsocconf]{llncs}
% Add the compsocconf option for Computer Society conferences.
%
% If IEEEtran.cls has not been installed into the LaTeX system files,
% manually specify the path to it like:
% \documentclass[conference]{../sty/IEEEtran}
\newcommand{\highlight}[1]{\colorbox{yellow}{#1}}


% Some very useful LaTeX packages include:

% *** MISC UTILITY PACKAGES ***
%
\usepackage{ifpdf}


% *** CITATION PACKAGES ***
%
%\usepackage{cite}


% *** MATH PACKAGES ***
%
\usepackage[cmex10]{amsmath}
\usepackage{amssymb}

% *** SPECIALIZED LIST PACKAGES ***
%
\usepackage{algorithmic}


% *** ALIGNMENT PACKAGES ***
%
\usepackage{array}

% *** SUBFIGURE PACKAGES ***
%
\usepackage[tight,footnotesize]{subfigure}

% *** FLOAT PACKAGES ***
%
\usepackage{fixltx2e}
\usepackage{stfloats}

% *** GRAPHICS PACKAGES ***
%
\usepackage{graphicx}

% *** BIBLIOGRAPHY PACKAGES ***
%
\usepackage{natbib}

% *** LANGUAGE PACKAGES ***
%
\usepackage[utf8]{inputenc} 
\usepackage[T1]{fontenc}      
\usepackage[francais]{babel}

% *** LAYOUT PACKAGES ***
%
\usepackage[top=4cm, bottom=4cm, left=4cm, right=4cm]{geometry}


\begin{document}
%
% paper title
% can use linebreaks \\ within to get better formatting as desired
\title{Le numérique dans\\l'accomplissement des SDGs}





% author names and affiliations
% use a multiple column layout for up to two different
% affiliations
% 
\author{Djavan Sergent \\
Master en Sciences Informatiques \\
djavan.sergent@etu.unige.ch}

\institute{Université de Genève}



% make the title area
\maketitle


\begin{abstract}
	<ABSTRACT>
\end{abstract}


\textbf{Mots clés} Citizen Science ; Monitoring ; Water Quality ; Sustainable Development


\section{Introduction}
	En 2000, les Nations-Unies lancent le programme des Millenim Developpment Goals (MDGs) qui s'étend jusq'en 2015 \cite{united_nations_millennium_2009}. Il s'agit d'un ensemble d'objectifs internationaux parmi lesquels on peut notamment citer l'éradication de l'extrême pauvreté et de la faim, combattre la mortalité infantile ou encore apporter une éducation à toutes et tous. Les 191 états membres des Nations-Unies ainsi que 22 organisations internationnales se sont engagées à participer activement à la réalisation de ces objectifs \cite{wikipedia_millennium_2017}.
	
	Fin 2015, beaucoup d'efforts ont été investis, mais les progrès sont encore très inégaux. Les différents pays membres des Nations-Unies ainsi que des organisations civiles se sont donc intéressées à l'agenda post-2015, c'est à dire aux objectifs futurs. Les Sustainable Developpment Goals (SDGs) ont étés acceptés comme relève des MDGs \cite{wikipedia_sustainable_2017}. Ceux-ci comportent 17 buts, chacuns subdivisé en objectifs. Les SDGs totalisent 169 objectifs qui possédent chacuns leurs propres indicateurs.
	\begin{figure}
		\begin{center}
			\includegraphics[width=300pt]{sdgs.png}
		\end{center}
		\caption{Icônes des SDGs (source : Sustainable Development Knowledge Platform)}
	\end{figure}
	\\
	Nous analysons dans cet article le rôle du numérique dans la réalisation des certains SDGs, particulièrement du point de vue de la participation citoyenne.

	\section{Sustainable Developpment Goals}
		\subsection{Objectifs}
		Nous nous intéressons, dans le cadre de cet article, aux buts décrits ci-dessous. Il est cependant important de noter que les objectifs sont intrinséquement liés entre eux et s'influencent mutuellement. Par exemple, en formant des citoyens à l'utilisation de matériel de mesure de qualité de l'eau on va agir non seulement sur la capacité à, entre autre, détecter la pollution mais également sur l'éducation.
		\begin{description}
			\item[ 3 - Good Health and Well-Being :] La santé et la pollution de l'air sont au centre de ce but.
			\item[ 6 - Clean water and sanitation :] Un accès universel à l'eau et aux installations sanitaires est essentiel pour la santé humaine, la prospérité économique et la préservation de l'environnement.
			\item[14 - Life below water :] L'acidification des océans, la surpêche ou encore la pollution marine ont un impact important sur la protection des océans. Leur dégradation provoque des effets sur certaines espèces marines mais également sur la biodiversité et le fonctionnement des écosystèmes.
		\end{description}		
					
		\subsection{Progrès, revue et indicateurs}
			Le High-Level Political Forum (HLPF), créé en 2012 à la suite de Rio20+, est chargé de promouvoir les objectifs, d'assurer un suivi et d'émettre des recommandations pour la réalisation des SDGs. Ce forum se réunit annuellement. \\
			Les dix-sept buts possèdent leurs propres indicateurs globaux et standardisés. Ces indicateurs servent à évaluer les progrès effectués au niveau international, national, régional et local. \\
			Le HLPF revoit une partie des objectifs lors des réunions annuelles, les autres sont abordés à une session ultérieure. La participation et la collaboration est revue chaque année pour favoriser le développement des SDGs.

\section{Monitoring environnemental}
	\subsection{Eau}
			L'eau est l'une des ressources naturelles les plus importante sur terre. Elle joue un rôle essentiel dans de multiples secteurs économiques, sanitaires et environnementaux.\\
			L'accès à l'eau potable, à des installations sanitaires et un plan de gestion des ressources est un enjeu majeur des SDGs. Réparties de façon inégale sur terre \cite{lefevre_repartition_nodate}, l'eau est essentielle pour le développement économique, l'agriculture, la protection de l'environnement ou encore la santé. Dans de nombreux pays tels que les États-Unis, la plus grande partie de cette eau est dédiée à l'agriculture \cite{world_business_council_for_sustainable_development_global_nodate}. Dans ce contexte, il est important de mettre en oeuvre des systèmes de gestion des ressources hydriques et de permettre un accès universel à des sources d'eau propre. Ce but a un impact sur d'autres tels que la santé ou la lutte contre la faim.\\
			Selon les rapports du secrétaire-général du conseil économique et social des Nations-Unies \cite{united_nations_economic_and_social_council_progress_2017}\cite{united_nations_economic_and_social_council_progress_2017-1}, un tiers de la population mondiale n'a, en 2015, pas accès à des installations sanitaires. Selon les même rapports, parmi eux, 946 millions n'ont accès à aucune infrastructure. La mauvaise gestion des déchêts humains représente un risque concret pour la santé et pour l'environnement \cite{ashbolt_microbial_2004}.\\
			Concernant l'accès à l'eau potable, la situation évolue positivement. On constate qu'en 2000, 82 pourcent de la population dispose d'une source d'eau aménagée contre 91 pourcent en 2015. Cependant, on estime également qu'environ 25 pourcent de la population mondiale est exposée à de l'eau contaminée par des matières fécales \cite{united_nations_goal_nodate-4}. \\
			Selon le rapport \cite{rana_water_2017}, il n'existe pas de plan complet qui permette la mise en place d'un système de gestion renouvelable des ressources en eau et le manque de données précises rend impossible l'évaluation des performances des approches actuellement implémentées. \\
			Toujours d'après ce même rapport, les innondations sont à l'origine de nombreuses maladies et dommages causés à des infrastructures. Les innondations peuvent causer des épidémies, comme le démontre le cas de Itaparica Dam au Brésil lors duquel, en 1988, plus de 2000 cas de gastroantérite sont déclarés sur une période de 42 jours. 88 d'entre eux s'avéreront mortels. Les innondations favorisent également la reproduction des moustiques et ainsi la propagation de maladies telles que la Rift Valley Fever (RFV) \cite{hanafi_rift_2010}.\\
			Un aspect également très important de la gestion de l'eau concerne les "Dead Zones" qui s'étendent de façon exponnentielle depuis 1960 \cite{diaz_spreading_2008}. Les "Dead Zones" sont de larges étendues d'eau qui n'ont plus assez d'oxygène pour assurer la vie marine. En 2008, 400 de ces zones affectant plus de 245'000km$^{2}$, majoritairement situées dans les océans, sont répertoriées. La cause principale du développement des "Dead Zones" est la libération dans l'eau de nutriments qui accélèrent la croissance de certaines algues, réduisant la quantité d'oxygène disponible sous la surface. Ces zones sont donc souvent situées à proximité de sites industriels et agricoles qui rejettent leurs nutriments dans l'eau. \\
			\begin{figure}
				\begin{center}
					\includegraphics[width=200pt]{marine-life-dead-zones.png}
				\end{center}
				\caption{Répartition des zones mortes (source : nasa.gov)}
			\end{figure} \\
			Autre constat : certains pays dépassent en consommation la quantité d'eau renouvelable à disposition et exercent donc une pression sur le cycle de l'eau. \\
			La multitude de problématiques présentées ci-dessus explique pourquoi l'eau est au coeur des SDGs. Depuis de très nombreuses années, une attention particulière est portée à la qualité de l'eau et aux dangers liés à une contamination de celle-ci \cite{ashbolt_microbial_2004}. Le manque d'infrastructures sanitaires dans les régions en développement les rends vulnérables aux morts par contamination de l'eau. Neuf morts sur dix touchent les enfants, et tous les décès surviennent dans ces régions. \\
			Aujourd'hui, de nombreux outils permettent de suivre avec précision les indicateurs classiques de qualité de l'eau tels que le taux d'oxygène, l'acidité, la température, la conductivité etc. Ceci est très important pour la réalisation du but 6.3 : <<By 2030, improve water quality by reducing pollution, eliminating dumping and minimizing release of hazardous chemicals and materials, halving the proportion of untreated wastewater and substantially increasing recycling and safe reuse globally>> \cite{united_nations_goal_nodate-4}. \\
		\subsection{Indice Biologique Global Normalisé}
			Depuis les années 1970, les auteurs de diverses études analysent les relations entre la faune et la qualité environnementale de leurs habitats. On considère en effet que les relevés sont capables de fournir des indicateurs sur l'état et la qualité de l'écosystème aquatique étudié. Dès lors, on a mis au point de nombreux outils basés sur les macro-invertébrés benthiques a des fins de diagnostiques concernant la qualité des écosystèmes aquatiques. <SOURCE IBGN>\\
			En Europe, les macro-invertébrés benthiques sont les éléments les plus utilisés pour comprendre, analyser et révéler les pressions anthropiques exercées car ils présentent les caractéristiques suivantes :
			\begin{itemize}
				\item Ils sontrelativement sédentaires et sont donc extrêment sensibles aux perturbations et pollutions importantes.
				\item Ils sont très hétérogènes, la plupart du temps abondants et disposent d'une grande variété de formes et d'espèces.
				\item Pour une variation environnementale donnée, il est très probable de trouver quelques organismes qui réagissent fortement et, souvent, de façon extrêmement rapide.
				\item La différence de caractéristiques entre les différentes organismes va modifier leur réponse en fonction de la nature et de l'intensité du stress.
				\item Leur durée de vie qui varie de quelques mois à quelques années est suffisante pour enregistrer et surveiller la qualité environnementale.
				\item On peut trouver des macro-invertébrés de façon abondante dans tous les types d'habitats.
				\item Ils sont relativement aisés à collecter et présentent l'avantage, par rapport au micro-organismes et au plancton, d'être bien plus facile à identifier.
			\end{itemize}
			L'IBGN est un des outils diagnistiques de la qualité des écosystèmes aquatiques qui utilise les macro-invertébrés benthiques pour leurs nombreuses caractéristiques. \\
			Le protocole de l'IBGN impose un échantillonage en 8 prélèvements réalisés sur des substrats différents, dans un ordre défini par la norme et en tenant compte de la vitesse du courant. On utilise un filet de type Surber d'une surface d' 1/20$^e$ de m$^2$ et donc les mailles mesurent 500>nano<m. On calcule l'indice en estimant différents paramètres en utilisant une liste finie de 152 taxons. Parmi ces 152 taxons, 38 seront considérés comme indicateurs et permettent la définition de 9 groupes faunistiques correspondant à leur pollusensibilité. Le calcul se fait alors en 3 étapes :
			\begin{enumerate}
				\item On détermine la "classe de variété taxonomique" (A) qui est égale au nombre de taxons récoltés sur les 152 potentiellement présents. Un taxon est considéré présent même s'il est représenté par un seul individu.
				\item On détermine le "groupe faunistique indicateur" (B) en ne prenant en compte que les taxons indicateurs. Il doit y avoir au moins 3 individus (ou 10 pour certains taxons) par échantillon.
				\item L'indice est obtenu par la formule suivante : $A + B - 1$
			\end{enumerate}
		\subsection{Air}
			La qualité de l'air est aujourd'hui responsable de nombreuses maladies, cancers et décès. Une étude chinoise récente met en avant l'augmentation de la fréquence de visite des hopitaux pour des problèmes respiratoires lors de l'augmentation du nombre de particules fines dans l'air \cite{liu_effects_2016}. La réduction du nombre de morts attribuables à la qualité de l'air est un des indicateurs de l'objectif 3 (Good-Health and Well-Being) \cite{united_nations_goal_nodate-5}. En opposition à la qualité de l'eau, celle de l'air s'est dégradée lors de la dernière décennie. En 2012, la pollution de l'air est responsable d'environ 5.5 milions de morts à travers le monde et environ la moitié de la population mondiale est exposée à un air dont la concentration en particules fines est supérieure à 10 microgrammes/m$^{3}$ \cite{yale_university_epi_2016}. Certains chercheurs affirment aussi qu'on peut constater un lien entre le nombre de tumeurs malignes du cerveau dans une région géographique et son taux de particules fines \cite{andersen_long-term_nodate}. \\
			La pollution de l'air impacte également le changement climatique en modifiant la composition de l'atmosphère et des océans. L'augmentation de la quantité de CO$_{2}$ dans l'air influe sur l'acidité des océans et impacte donc les espèces marines sensibles à ces changements. La modification de la composition chimique des océans impacte directement la reproduction de certaines espèces marines et la capacité des espèces de cnidères, tels que les coraux, à créer leur exosquelette \cite{hoegh-guldberg_coral_2007}.  \\
	
	\section{Monitoring sociétal}
		\subsubsection{Santé}
		\subsubsection{Sécurité}
		\subsubsection{Développement}
		
\section{Participation citoyenne}
	Les citoyens peuvent contribuer à la réalisation de ces SDGs par plusieurs moyens, qu'il s'agisse de la récupération, du traitement des données receuillies sur le terrain, de la mise à disposition de ressources ou d'outils. On peut citer en exemple le "World Water Monitoring Day", journée lors de laquelle des miliers de citoyens vont récupérer proche de chez eux des échantillons d'eau dont la qualité est analysée. \\
	Lorsque l'étendue géographique à couvrir et/ou la durée des observations dans le temps sont des paramètres importants, il est intéressant de recourir à un nombre important de citoyens bénévoles non spécialistes plutôt qu'à un petit groupe d'expert. De plus, leur nombre et leur répartition sur le terrain permet de limiter les risques financiers, de biais et d'évaluation non-neutre. Des protocoles standardisés sont nécessaires dans cette approche citoyenne.
		INatrualist, NatureBytes, Epicollect, SeeClickFix, Water Reporter, Project Noah		

	\subsection{Projets}
		\subsubsection{Aqueduct Water Risk Atlas}
		Le projet Aqueduct Water Risk Atlas a pour ambition de créer une carte en ligne représentant les risques hydriques au niveau mondial selon 12 indicateurs répartis en 3 catégories. Ce projet s'adresse tout autant à des entreprises qu'aux gouvernement afin de comprendre les enjeux et risques liés à l'eau.
		\subsubsection{InfoAmazonia}
		InfoAmazonia regroupe de nombreux projets autour de la forêt tropicale amazonienne. Les informations recueillies par sont ensuite rendues disponibles via une carte interactive. Celle-ci présente différentes vues, telles que, entre autres, les zones protégées, les zones de minages ou encore de déforestation. Les citoyens ayant des informations à diffuser sont invités à contribuer au projet au moyen d'un formulaire qui permet de soumettre sa propre histoire ou des données. Le projet InfoAmazonia est soutenu par plusieurs acteurs locaux. \\
		Parmi les différentes sources de données du projet, on peut notamment citer Mare d'Agua, un matériel ouvert qui permet de surveiller la qualité de l'eau. un réseau de capteurs sans fils (WSN) qui récupère les informations relatives à la qualité de l'eau telles que la conductivité électrique, le pH, le potentiel d'oxyréduction (OPR), la température, la pression barométrique ainsi que la température du capteur. Cet équipement peut être utilisé pour surveiller la qualité de l'eau dans des boîtes à eau, des citernes et réservoirs ainsi que les corps d'eau à faible débit. Inspiré du capteur "Riffle", ce matériel a été développé en collaboration avec Public Lab (réseau américain de science citoyenne) ainsi que Dev Technologia, une startup de Sao Paulo University. Les données recueillies sont envoyées à un serveur via le réseau de téléphonie mobile. Les relevés sont effectués à une fréquence d'un par heure. Il est nécessaire de nettoyer régulièrement les capteurs afin d'éviter que les relevés ne soient erronés à cause de l'accumulation de débris. Il s'agit d'un projet open source et open hardware, les fichiers et firmware sont disponible sous licence MIT. \\
		L'équipement est modulaire et comporte 3 modules :\\
		\begin{enumerate}
			\item Le premier module est un Arduino Mega et gère le traitement des données ainsi que l'alimentation en énergie depuis une source externe.
			\item Le deuxième module est une coque Arduino conçue grâce à la plateforme Eagle. Cette coque dispose d'un capteur de pression barométrique et de connecteurs permettant d'ajouter les capteurs de pH, température, OPR et de conductivité. Cette conception modulaire permet d'optimiser le rapport coût-bénéfices en fonction de l'application projetée.
			\item Le troisième module permet la communication au travers du réseau téléphonique mobile pour envoyer les données au serveur.
		\end{enumerate}
		Le projet est soutenu par Google et par des institutions et entreprises locales.
		\subsubsection{World Water Monitoring Day}
		\subsubsection{World Water Monitoring Challenge}
		Le WWMC est un projet du EarthEcho International. Cet évènement est conduit chaque année entre mars et décembre réunit plus de 1'500'00 participants dans 146 pays. L'idée est d'équiper et former les citoyens afin qu'ils surveillent leurs points d'eau locaux tout autour du monde, avec pour le moment 77'685 endroits surveillés. La sensibilisation et l'incitation à la surveillance pour protéger les ressources hydriques sont les points clés de ce projet. \\
		Les citoyens participent par plusieurs voies :
		\begin{itemize}
			\item Par la surveillance des points d'eau à l'aide de kits.
			\item Par le partage des données au travers d'une plateforme et des réseaux sociaux.
			\item Par l'utilisation des informations pour protéger les ressources hydriques vulnérables. 
		\end{itemize}
		Les kits comprennent : 1 livre d'instruction, 1 récipient pour les échantillons, 1 tube de à essais de pH, 1 fiole de d'oxygène dissous, 2 bandes de température, 50 comprimés de réactif au pH, 100 comprimés de réactif d'oxygène dissous, 1 disque de Secchi et 1 nuancier pour l'interprétation des résultats (DO, pH, turbidité). Les kits sont conçus pour permettre d'effectuer 50 tests et coûtent environ 50 dollars. Le projet permet également à certaines organisations, sous conditions, d'obtenir gratuitement des kits.
		\subsubsection{Riverfly Monitoring Initiative}
		Ce projet, réalisé aux Royaumes-Unis, réunis de nombreux acteurs tels que pêcheurs, authorités compétentes, scientifiques et entomologistes dans le but de protéger la qualité de l'eau des rivières, de mieux comprendre les populations des mouches de rivières et de conserver leur habitats. Ces insectes jouent un rôle fondamental dans les rivières, en particulier concernant l'alimentation de certains poissons et chauves-souris car ils font partie du plancton aérien. Ces insectes, apparus au Carbonifère (il y a environ 300 millions d'années) sont extrêmement sensibles à la pollution lumineuse et chimique, notamment par les pesticides. On compte parmi ces espèces les éphémères, les plécoptères et les trichoptères. Ils passent la plus grande partie de leur vie au stade larvaire, dans l'eau. De ce fait, les facteurs tels que le débit d'eau, sa qualité ou encore son niveau ont un impact majeur sur ces espèces. Leurs caractéristiques les rendent particulièrement pertinentes en tant qu'indicateurs lors d'études de la qualité de l'eau. Par exemple, lors du rejet d'eaux usées, un des bio-indicateurs de pollution peut être la population de certaines espèces de plécoptères puisque celles-ci, sensibles à la quantité d'oxygène dissous, vont chuter brusquement.\\
		
		\subsubsection{Restoration Assessment Initiative}
		\subsubsection{Homebrew Sensing Project}
		\subsubsection{Open Water Project}
		\subsubsection{Open Air}
		\subsubsection{Open Land}
		\subsubsection{etc...}

\section{Conclusion}\label{sec:conclusion}

% use section* for acknowledgement
\section*{Remerciements}


\bibliographystyle{alpha}
\bibliography{soa} %%% soa.bib is the file containing bibliographic entries


% that's all folks
\end{document}


